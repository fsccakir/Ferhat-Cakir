% Options for packages loaded elsewhere
\PassOptionsToPackage{unicode}{hyperref}
\PassOptionsToPackage{hyphens}{url}
\PassOptionsToPackage{dvipsnames,svgnames,x11names}{xcolor}
%
\documentclass[
  12pt,
]{article}
\usepackage{amsmath,amssymb}
\usepackage{iftex}
\ifPDFTeX
  \usepackage[T1]{fontenc}
  \usepackage[utf8]{inputenc}
  \usepackage{textcomp} % provide euro and other symbols
\else % if luatex or xetex
  \usepackage{unicode-math} % this also loads fontspec
  \defaultfontfeatures{Scale=MatchLowercase}
  \defaultfontfeatures[\rmfamily]{Ligatures=TeX,Scale=1}
\fi
\usepackage{lmodern}
\ifPDFTeX\else
  % xetex/luatex font selection
\fi
% Use upquote if available, for straight quotes in verbatim environments
\IfFileExists{upquote.sty}{\usepackage{upquote}}{}
\IfFileExists{microtype.sty}{% use microtype if available
  \usepackage[]{microtype}
  \UseMicrotypeSet[protrusion]{basicmath} % disable protrusion for tt fonts
}{}
\makeatletter
\@ifundefined{KOMAClassName}{% if non-KOMA class
  \IfFileExists{parskip.sty}{%
    \usepackage{parskip}
  }{% else
    \setlength{\parindent}{0pt}
    \setlength{\parskip}{6pt plus 2pt minus 1pt}}
}{% if KOMA class
  \KOMAoptions{parskip=half}}
\makeatother
\usepackage{xcolor}
\usepackage[margin=1in]{geometry}
\usepackage{longtable,booktabs,array}
\usepackage{calc} % for calculating minipage widths
% Correct order of tables after \paragraph or \subparagraph
\usepackage{etoolbox}
\makeatletter
\patchcmd\longtable{\par}{\if@noskipsec\mbox{}\fi\par}{}{}
\makeatother
% Allow footnotes in longtable head/foot
\IfFileExists{footnotehyper.sty}{\usepackage{footnotehyper}}{\usepackage{footnote}}
\makesavenoteenv{longtable}
\usepackage{graphicx}
\makeatletter
\def\maxwidth{\ifdim\Gin@nat@width>\linewidth\linewidth\else\Gin@nat@width\fi}
\def\maxheight{\ifdim\Gin@nat@height>\textheight\textheight\else\Gin@nat@height\fi}
\makeatother
% Scale images if necessary, so that they will not overflow the page
% margins by default, and it is still possible to overwrite the defaults
% using explicit options in \includegraphics[width, height, ...]{}
\setkeys{Gin}{width=\maxwidth,height=\maxheight,keepaspectratio}
% Set default figure placement to htbp
\makeatletter
\def\fps@figure{htbp}
\makeatother
\setlength{\emergencystretch}{3em} % prevent overfull lines
\providecommand{\tightlist}{%
  \setlength{\itemsep}{0pt}\setlength{\parskip}{0pt}}
\setcounter{secnumdepth}{5}
\usepackage{polyglossia}
\setmainlanguage{turkish}
\usepackage{booktabs}
\usepackage{caption}
\captionsetup[table]{skip=10pt}
\ifLuaTeX
  \usepackage{selnolig}  % disable illegal ligatures
\fi
\IfFileExists{bookmark.sty}{\usepackage{bookmark}}{\usepackage{hyperref}}
\IfFileExists{xurl.sty}{\usepackage{xurl}}{} % add URL line breaks if available
\urlstyle{same}
\hypersetup{
  pdftitle={Türkiye'de Eğitim ve Yoksulluk ilişkisi},
  pdfauthor={Ferhat Sertaç Çakır},
  colorlinks=true,
  linkcolor={Maroon},
  filecolor={Maroon},
  citecolor={Blue},
  urlcolor={blue},
  pdfcreator={LaTeX via pandoc}}

\title{Türkiye'de Eğitim ve Yoksulluk ilişkisi}
\author{Ferhat Sertaç Çakır\footnote{Ferhat Sertaç Çakır, \href{https://github.com/fsccakir/Ferhat-Cakir.git}{Github Repo}}}
\date{}

\begin{document}
\maketitle
\begin{abstract}
Ham veri üzerinde değişkenlerin adlarında yaptığım düzenlemeler ile başladığım analize önce özet istatistik tablosu çıkararak değişkenler hakkında bilgi verdim. Daha sonrasında nüfusun eğitim düzeyini gösteren değişkenler arasında histogram, dağılım ve çizgi grafikleri oluşturdum. Bu grafiklerde eğitim düzeyleri ile yoksulluk ilişkisini ele aldım. Bu ilişkide lise dengi okul veya yükseköğretim mezunu gibi eğitim düzeylerinde yoksul sayısının okur-yazar olmayanlara veya düşük eğitim seviyesinde olanlara göre önemli ölçüde fark olduğunu gözlemledim. Bu gözlemlere göre eğitim seviyesi düşük olanın yoksulluk sayısında yüksek olduğunu ortaya koyuyordu. Bu farklar eğitim düzeyleri arttıkça daha da artmaktadır. En sonunda grafiklerden gözlemleyebildiğimiz eğitim ve yoksulluk arasındaki ilişkinin yönünü net olarak ortaya koymak için değişkenler arasından okur-yazar olmayan ile yükseköğretim eğitim düzeyindekilerinin yoksul sayıları arasında regresyon analizi yapmaya çalıştım. Bu analiz sonucu denklem katsayısına göre bu değişkenler arasında ters yönlü bir ilişki olduğu belirlenmiş oldu. Yani eğitim düzeyi arttıkça yoksul sayısının düştüğünü, araştırma konumuza dair verdiğimiz sorunun cevabını ulaştığımızı söyleyebilirim.
\end{abstract}

\hypertarget{giriux15f}{%
\section{Giriş}\label{giriux15f}}

Tüm Dünya'da olduğu gibi Türkiye'de de yoksulluk başlıca sorunlardandır. Bununla beraber son yıllarda hızla artan yoksulluk ile birlikte hem ulusal hem de uluslararası düzeyde yoksulluğu önlemek amacıyla çeşitli politikalar uygulanmaktadır. Bunların en başında gösterebileceğimiz eğitim politikaları yer almaktadır.Bu yüzden proje önerimin araştırma konusu olarak da Türkiye'de eğitim ve yoksulluk arasında ne tür bir ilişki olduğunu incelemeyi seçtim. Proje önerimin veri setini Türkiye İstatistik Kurumu'nun web sitesinden elde ettim. Veri setinde yoksul sayısı ve yoksulluk oranı başlıkları altında Türkiye'de nüfusun eğitim düzeyleri ve okur-yazar olma durumları gibi değişkenler bulunmaktadır. Bununla birlikte yoksulluk riskleri de veri setinde yer almaktadır. Bu veriler 2006-2021 yılları arası incelenmiş olup 50 gözleme dayanmaktadır.

\hypertarget{uxe7alux131ux15fmanux131n-amacux131}{%
\subsection{Çalışmanın Amacı}\label{uxe7alux131ux15fmanux131n-amacux131}}

Çalışma, Türkiye'de yoksulluğun ve eğitimin birbirleri arasındaki ilişkisini, bu ilişkinin ne tür bir ilişki olduğunu ortaya koymayı ve analiz edebilmeyi konu edinmiştir.

Yoksulluk ve eğitim üzerine birçok çalışma bulunmaktadır. Bu çalışma, bu konu ile ilgili yayınlanmış makaleleri, kitapları, grafikleri ve diğer verileri de inceleyip analize dahil edebilmeyi hedeflemektedir. Bununla birlikte Türkiye gibi gelişmekte olan ülkeleri ve diğer Avrupa ülkelerini de inceleyip veri setinin ortaya koymuş olduğu eğitimin yoksulluğu azaltmadaki etkin rolünü gösterebilmek ve desteklemektir. Bunun sonucunda konu ile ilgili bulunan veriler ışığında bu ilişkiyi incelemek ve istatistiksel analiz yapabilmeyi amaçlamaktadır.

\hypertarget{literatuxfcr}{%
\subsection{Literatür}\label{literatuxfcr}}

Eğitim ve yoksulluk araındaki ilişkiyi incelemeden önce kısaca tanımlarını yapmamız gerekir.Yoksulluk mutlak yoksulluk ve göreli yoksulluk olarak ikiye ayrılır. Mutlak yoksulluk, insan varlığının devamı için gerekli olan yiyecek, içecek ve barınma gibi en temel ihtiyaçlarının karşılanamaması durumudur. Göreli yoksulluk ise, insanların temel ihtiyaçlarını karşılayabilmesine rağmen kişisel kaynaklarının yetersizliği nedeniyle toplumun genel refah düzeyinin altında kalması durumudur. Eğitim, her ne kadar görüş birliğine varılan tek bir tanımı olmasa da eğitimi genel olarak, kültürün genç kuşaklara aktarılması ve toplumun varlığını sürdürebilmesi için gerekli sosyal değişimleri yapabilecek yaratıcı kişilerin yetiştirilmesi olarak tanımlamak mümkündür.

Literatür incelendiğinde eğitimin yoksulluk üzerinde dramatik bir etkiye sahip olduğu görülmektedir\\
Yoksulluğa sebep olan birçok faktör mevcuttur. Bunlar arasında ön plana çıkanlar ise; ekonomik, işsizlik, göç, cinsiyet ve de eğitimdir. Yoksulluk olgusu tüm yönleriyle incelendiğinde, en önemli nedenin eğitimden kaynaklandığı söylenebilir. Bu bağlamda eğitimin ekonomi ve işsizlikle doğrudan ilişkili olduğu ifade edilebilir.Bu bağlamda söz konusu bireylerin yoksulluğun kısır döngüsüne girmelerine engel olma noktasında alınacak tedbirler ve uygulanacak politikalar hayati önem taşımaktadır. Yoksulluğun nesilden nesille bir miras olarak bırakılmamasın tek yolu ise eğitim seviyesinin yükseltilmesinden geçmektedir
Öte yandan eğitim, insanların ekonomiye ve topluma katılma becerileri kazanmalarına yardımcı olur.Bulguları göz önüne alındığında, eğitim özünde geliri eşit dağıtmak için bir mihenk taşıdır ve yoksullara ekonomik büyümeden daha fazla yararlanma fırsatı sağlar.Bu nedenle eğitim politikaları, örgün ve yaygın eğitime katılımı artırmaya çalışırken, yoksullukla mücadelede önemli bir rol üstlenmeyi hedeflemeli

Sonuç olarak, yoksulluk ve eğitim birbiriyle ters orantılıdır.
Aynı zamanda eğitimden yoksulluğa doğru ters nedensellik olabileceği gibi, yoksulluktan eğitime doğru da eşit derecede nedensellik olabilir. Örneğin, eğitime yapılan yatırım, insanların üretkenliğinin yanı sıra ücretleri veya geliri artırarak yoksulluğu azaltır. Ayrıca eğitim, insanların daha etkin üretim yapma kapasitelerini geliştiren bazı gerekli becerileri edinmelerini sağlar. Öte yandan yoksulluk, öğrencilere sağlanan kaynakları etkileyerek eğitimin kalitesini ve eğitime eşit erişimi kısıtlamaktadır

Yoksulluk nedeniyle bireylerin yetersiz eğitim alması bir yandan yoksulluğu devam ettirici bir etki yaratırken; bir diğer yandan bu bireylerin topluma katkılarının da düşük düzeyde kalmasına neden olarak ülkenin kalkınmasını olumsuz yönde etkilemektedir. Bu nedenle yoksullukla mücadelede eğitim konusu sosyal politika alanında ve ekonomik kalkınmada önemli bir unsuru teşkil etmektedir.

\hypertarget{veri}{%
\section{Veri}\label{veri}}

Veri setimi TÜİK'in Gelir ve Yaşam Koşulları adlı araştırma çalışmasındaki istatistiki tablolardan edindim. Veri setimde Türkiye'deki nüfusun okur-yazar olma, lise veya dengi, yükseköğretim mezunu gibi eğitim durumlarını gösteren değişkenlerle beraber yoksulluk risklerini gösteren yoksul sayıları ve oranları bulunmaktadır. Değişkenlerin isimlerini özet tabloya ve grafiklere aktarmadan önce clean-names fonksiyonu özel karakterleri düzenledim. Sonrasında ise rename with fonksiyonu özel karakterlerde tekrardan bir düzenlemeye giderek değişkenlerde daha sade bir adlandırma yaptım ve özet istatistikleri içeren tabloya aktardım.

Özet istatistiksel tabloya baktığımızda araştırma sorumu da destekleyen veriler bulunmaktadır. Tablodan eğitim ve yoksulluk arasındaki ilişkide eğitim düzeyi arttıkça yoksul sayısının da düştüğünü görebiliyoruz. Mesela eğitim düzeyinin ilk seviyesi olan illiterate olan okur-yazar olmayanın ile son seviyesi highereducation olan yükseköğretimi kıyaslarsak bu farkı net bir biçimde görebiliyoruz. illiterate4 değişkeninin yoksul sayısıdanki ortalaması 1912 iken highereducation\_8 değişkeninin 200. Bu eğitim ve yoksulluk arasındaki ilşkiyi bir kez daha ortaya açıkça koyuyor. Keza yine highschool or equivalent ile literate with no degree değişkenlerini incelediğimizde aynı sonuca ulaşabiliyoruz.

\begin{table}[ht]
\centering
\caption{Özet İstatistikler} 
\label{tab:ozet}
\begin{tabular}{lccccc}
  \toprule
 & Ortalama & Std.Sap & Min & Medyan & Mak \\ 
  \midrule
highereducation\_14 & 2.26 & 1.16 & 1.00 & 2.00 & 5.00 \\ 
  highereducation\_8 & 187.91 & 154.06 & 24.00 & 134.00 & 548.00 \\ 
  highschool\_or\_equivalent\_13 & 8.00 & 2.50 & 4.00 & 8.00 & 13.00 \\ 
  highschool\_or\_equivalent\_7 & 859.15 & 353.38 & 424.00 & 811.00 & 1804.00 \\ 
  illiterate10 & 32.29 & 5.17 & 24.00 & 31.50 & 42.00 \\ 
  illiterate4 & 1900.74 & 366.68 & 1203.00 & 1870.00 & 2571.00 \\ 
  lessthan\_high\_school & 4903.41 & 1100.69 & 3392.00 & 4799.00 & 6486.00 \\ 
  lessthan\_high\_school\_2 & 16.76 & 3.72 & 12.00 & 16.50 & 21.00 \\ 
  literatewith\_no\_degree\_11 & 29.74 & 5.14 & 22.00 & 29.50 & 37.00 \\ 
  literatewith\_no\_degree\_5 & 1134.56 & 203.24 & 823.00 & 1138.50 & 1471.00 \\ 
   \bottomrule
\end{tabular}
\end{table}

\hypertarget{yuxf6ntem-ve-veri-analizi}{%
\section{Yöntem ve Veri Analizi}\label{yuxf6ntem-ve-veri-analizi}}

Bu bölümde araştırma konusunu daha net ele alabilmek için grafikler ve analizler yaptım. Öncelikle yoksulluk ile eğitim arasındaki ilişkiyi inceleme üzere yükseköğretim ve okur-yazar olmayanların yoksul sayıları arasındaki ilişkiyi görmek için bu nokta grafiği oluşturdum.. Yükseköğretim yoksul sayıları 200-600 arasında iken okur-yazar olmayanlarda 1500-2500 arasında değişmektedir. Bu da eğitim seviyesinin en yüksek ve en düşük seviyesinde yoksulluk riskini bir ölçüde bize göstermektedir. Daha sonrasında lise altı eğitim ile yükseköğretim değişkenlerinin histogramını oluşturdum. Devamında lise dengi okul ve bir okul bitirmeyenlerin yoksulluk oranları arasında nokta grafiğini oluşturdum. Bununla beraber okur-yazar olmayan ile lise altı eğitimde olanlar arasında çizgi grafiği oluşturdum. Daha sonrasında yükseköğretim ile okur-yazar olmayanların yoksul sayıları arasındaki dağılım, korelasyon ve diğer ilişkileri gösteren bir diyagram oluşturdum. Bu oluşturduğum diyagramlara ve grafiklere baktığımızda eğitim seviyesi ve yoksulluk arasındaki ilişkinin ters yönlü bir ilişki olduğu ortak sonucuna ulaşabiliriz. Devamında bu ilişkiyi daha iyi analiz edebilmek için yükseköğretim ile okur-yazay olmayanların yoksul sayıları arasında regresyon analizine başvurdum ve scoremod fonksiyonu ile özet bir tablo oluşturudum. Regresyon analizi sonucu bulduğum ``illiterate4 = 2043.3612 + highereducation\_8 . -0.7012'' sonuç ile bu ilişkinin yönünü teyit etmiş oldum.

\begin{center}\includegraphics{final_files/figure-latex/unnamed-chunk-4-1} \end{center}

\begin{center}\includegraphics{final_files/figure-latex/unnamed-chunk-4-2} \end{center}

\begin{center}\includegraphics{final_files/figure-latex/unnamed-chunk-4-3} \end{center}

\begin{center}\includegraphics{final_files/figure-latex/unnamed-chunk-4-4} \end{center}

\begin{center}\includegraphics{final_files/figure-latex/unnamed-chunk-4-5} \end{center}

\begin{center}\includegraphics{final_files/figure-latex/unnamed-chunk-4-6} \end{center}

\begin{center}\includegraphics{final_files/figure-latex/unnamed-chunk-4-7} \end{center}

\includegraphics{final_files/figure-latex/unnamed-chunk-5-1.pdf}
Call:
lm(formula = illiterate4 \textasciitilde{} highereducation\_8, data = cy)

Residuals:
Min 1Q Median 3Q Max
-623.81 -305.00 40.98 305.55 596.47

Coefficients:
Estimate Std. Error t value Pr(\textgreater\textbar t\textbar)\\
(Intercept) 2034.1929 96.9997 20.971 \textless2e-16 ***
highereducation\_8 -0.7102 0.4016 -1.769 0.0865 .\\
---
Signif. codes: 0 `\emph{\textbf{' 0.001 '}' 0.01 '}' 0.05 `.' 0.1 ' ' 1

Residual standard error: 355.4 on 32 degrees of freedom
Multiple R-squared: 0.08905, Adjusted R-squared: 0.06058
F-statistic: 3.128 on 1 and 32 DF, p-value: 0.08649

{[}1{]} ``illiterate4 = 2034.1929 + highereducation\_8 . -0.7102''

\hypertarget{sonuuxe7}{%
\section{Sonuç}\label{sonuuxe7}}

Sonuç olarak projeyi, veriyi, grafikleri, tabloları, diagramları, özet istatistikleri ve yapılan diğer çalışmaları özetlersek eğer başlangıçta sorduğumuz ve araştırmaya çalıştığım yoksulluk ile eğitim seviyesi arasındaki ilişkinin ne tür bir ilişki olduğu sorusuna, analizlere dayanarak bu ilişkinin ters yönlü bir ilişki olduğu sonucuna vardığımı söyleyebilirim. Yani eğitim seviyesi arttıkça yoksulluk riskinin azaldığını söyleyebilirim. Bu çalışmayı desteklemek için Avrupa ülkeleri için de kıyaslayıp, Avrupa ülkelerindeki sonuçlar ile birlikte bir karşılaştırma yaparak geliştirebiliriz.

\end{document}
